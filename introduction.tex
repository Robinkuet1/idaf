\section{Einführung}
Test\autocite{HUNDRET-YEARS-CPI}

\subsection{Definition}

\subsubsection{Konsumenten Preis Index}


Ein Konsumenten Preis Index ist der meisstverbreitete Weg um die Inflation zu berechnen.
Dieser Index beinhaltet ein vordefinierter Katalog aus Preisen der wichtigsten Gütern und Dienstleistungen.
Der Katalog beinhaltet Gruppen von Gütern, die nach ihrer Wichtigkeit für einen durchschnittlichen Menschen gewichtet sind.
Einzelnen Gruppen können neu gewichtet um sich der stets wechselnden Bedürfnisen anzupassen.
% TODO Vor/Nachteile
Die Jährliche Inflationsrate kann mit folgender Formel berechnet werden:
$$Inflation = \frac{CPI_{Jahresanfang} - CPI_{Jahresend}}{CPI_{Jahresanfang}} \times 100$$

\subsubsection{Personal Consumption Expeditures}

\subsection{Geschichte}

\subsubsection{USA - CPI}
Bereits vor dem offiziellen "Consumer Price Index" wurden in den Vereinigten Staten regelmässige Statistiken zur Inflation publiziert.
Da diese anfänglichen Statistiken standen viel unter öffentlicher Kritik,
da sie grösstenteils auf dem Preis von Nahrungsmittel basierten und dies zum Teil wenig Bezug zur allgemeine Inflation hat.
Währent und nach dem ersten Weltkrieg wurden aufgrund hoher Inflation vermehrt umfangreichere Statistiken erstellt, die neben dem Preis von Nahrungsmittel auch noch Kleidung, Miete, Benzin, Elektrizität und weiteres aufgenommen.
Der CPI, dazumals als "Cost of Living Index" bezeichnet, wurde 1921 als offizieller, nationaler Index erstellt und seither jährlich vom "Bureau of Labor Statistics" veröffentlicht.
Basierend auf anderen Statistiken oder Schätzungen wurden Daten von 1913 an im Index einbezogen.
Die Gewichtung der einzelnen Gruppen wurde basierend auf Resultaten von Umfragen erstellt und seit der Einführung des Indexes mehrmals angepasst.

\begin{figure}[htb]
	\centering
	\begin{tikzpicture}
		\begin{axis}[
				date coordinates in=x,
				xmin=1913-01-01,
				xmax=2024-01-01,
				xtick distance=365.3 * 10,
				xticklabel style={
						rotate=90,
						anchor=near xticklabel,
					},
				xticklabel=\year,
				title={Inflation Data},
				line width=1pt,
				legend style={at={(0.03,0.9)},anchor=west}
			]
			\legend{USA}
			\addplot[color=blue!50!cyan,smooth,tension=0.7,very thick] table [col sep=semicolon,x=Date,y=USA] {data/inflation.csv};
		\end{axis}
	\end{tikzpicture}
	\caption{Caption}
	\label{fig:usa}
\end{figure}


