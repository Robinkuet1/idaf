\section{USA}


\subsection{Bürgerkrieg}
Auch wie für andere Währungen üblich, in der Zeit, basierte auch der US Dollar anfangs auf dem Gold Standart.
Dies bedeutet, dass für alles zirkulierende Geld, eine equivalent wertvolle Menge an Gold oder Silber im Besitz der USA existieren muss.
Dazumals war dies umgesetzt indem die Münzen aus Silber und Gold hergestellt wurden. % TODO check
Währent dem Bürgerkrieg konnte auf beiden Seiten, aufgrund der Kriegskosten, eine sehr Hohe Inflation festgestellt werden.
In den Vereinigten Staaten wurde der Gold Standart temporär abgeschafft, um der Regierung mehr möglichkeiten zur Bekämpfung der Inflation und der Rückzahlung von Schulden zu ermöglichen.

Den Burgerkrieg zu finanzieren war für beide Seiten eine grosse Herausforderung, bei der die Bevölkerung mit sehr hoher Inflation zu kämpfen hatte.
Ein Grossteil des Krieges wurde durch das Herausgeben von neuem Geld und dem Ausleihen von Geld ermöglicht.
Auf Seiten der Konfederation, wurde über 60\% der Kriegskosten durch das Drucken von Geld bezahlt.
Auch bei den Vereinigten Staaten wurde der Gold Standart temporär abgeschaft um neues Geld zu drucken.
Jedoch wurde mit der Inflation nur ca. 13\% der Kreigskosten bezahlt.
Auf Seiten der Vereinigten Staaten wurden eininge neue Massnahmen ergriffen um mit der Inflation umzugehen.
Dazu gehörten eine Vermögenssteuer und Staatliche Leihen, die an die Bevölkerung verkauft wurden.


\subsection{Erster Weltkrieg}
Währent den 1910 Jahren erreichte die Inflation ein neues Hoch von über 20\% Jährlicher Inflation, die bis heute nicht geschlagen wurde.
Diese ist grossteils auf den Ersten Weltkrieg zurückzuführen.
In 1918 wurden bis zu 20\% des GDP der USA für den Krieg verwendet.
Die Preise einiger Güter wurden vom Staat kontrolliert und Lebensmittel wurden teilweise rationiert.
Ein Grossteil diese Krieges wurde durch Steuererhöhungen und Leihen in Form von "Liberty Bonds" ermöglicht.
Auch nach dem Krieg blieb die Inflation noch über normalem Niveau und legte bis 1920 nochmals zu.
Die Amerikanische Wirtschaft befand sich immer noch in einer expansiven Phase, doch eine Rezession und Deflation wurde von der Allgemeinheit erwartet.
Diese trat dann Mitte 1920 abrupt auf und bis im Januar 1921 war die Jährliche Inflationsrate von fast 20\% auf -10\% gesunken.
% TODO Rezession + Goldenen Zwanziger Jahre

\subsection{Wirtschaftskrise 1929}
Auf die Goldenen Zwanziger Jahre, eine Zeit wirtschaftlicher Expansion folgte am %TODO Datum 1929 die bekannte Wirtschaftskrise.
%TODO Irgendöpis über d Wirtschaftskrise

Auf diese Wirtschaftskrise folgte gegen Ende 1930 eine Deflation von bis zu 10\% pro Jahr.
Diese Periode wird viel als Start eine massiven Rezession, die unter dem Name "The Great Depression" bekannt ist.
Trotz hoher Arbeitslosigkeit und schwacher wirtschaftlicher Leistung blieb die Inflation bis 1937 im positiven Bereich, was für Rezessionen eher ungewöhnlich ist. %TODO check
%TODO meh über d Rezession

\subsection{Zweiter Weltkrieg}
Bereits vor dem Kriegseintrit der USA begann die Inflationsrate zu Steigen.
Dies ist zum Teil auf die militärische Aufrüstung zurückzuführen. %TODO check
Mit dem Kriegseintrit der USA in den 2. Weltkrieg stieg die Inflationsrate auf ein Niveau, dass seit dem Ersten Weltkrieg nicht mehr gesehen war.
Bereits 1942, ein Jahr nach dem Kriegseintrit der USA erreichte die Inflation den Hochstand währent dem Zweiten Weltkrieg.
Währent dem Krieg wurden vielfach Preise von der neu gegründeten staatlichen Institution "Office of Price Administration" kontrolliert.
Dies hatte grösstenteils sehr positive Auswirkungen auf die Wirtschaft und dadurch auch auf die Inflation. %TODO check
% TODO Auswirkungen auf die Wirschaft 

\subsection{Korea Krieg}


\subsection{Vietnam Krieg}

\subsection{Banken Krise 2008}

\subsection{Corona Pandemie}
